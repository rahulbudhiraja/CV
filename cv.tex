%%%%%%%%%%%%%%%%%%%%%%%%%%%%%%%%%%%%%%%%%%%%%%%%%%%%%%%%%%%%%%%%%%%%%%%%
%%%%%%%%%%%%%%%%%%%%%% Simple LaTeX CV Template %%%%%%%%%%%%%%%%%%%%%%%%
%%%%%%%%%%%%%%%%%%%%%%%%%%%%%%%%%%%%%%%%%%%%%%%%%%%%%%%%%%%%%%%%%%%%%%%%

%%%%%%%%%%%%%%%%%%%%%%%%%%%%%%%%%%%%%%%%%%%%%%%%%%%%%%%%%%%%%%%%%%%%%%%%
%% NOTE: If you find that it says                                     %%
%%                                                                    %%
%%                           1 of ??                                  %%
%%                                                                    %%
%% at the bottom of your first page, this means that the AUX file     %%
%% was not available when you ran LaTeX on this source. Simply RERUN  %%
%% LaTeX to get the ``??'' replaced with the number of the last page  %%
%% of the document. The AUX file will be generated on the first run   %%
%% of LaTeX and used on the second run to fill in all of the          %%
%% references.                                                        %%
%%%%%%%%%%%%%%%%%%%%%%%%%%%%%%%%%%%%%%%%%%%%%%%%%%%%%%%%%%%%%%%%%%%%%%%%

%%%%%%%%%%%%%%%%%%%%%%%%%%%% Document Setup %%%%%%%%%%%%%%%%%%%%%%%%%%%%

% Don't like 10pt? Try 11pt or 12pt
\documentclass[10pt]{article}

% This is a helpful package that puts math inside length specifications
\usepackage{calc}
%\usepackage{graphix}
\usepackage{graphics}
% Simpler bibsection for CV sections
% (thanks to natbib for inspiration)
\makeatletter
\newlength{\bibhang}
\setlength{\bibhang}{1em}
\newlength{\bibsep}
 {\@listi \global\bibsep\itemsep \global\advance\bibsep by\parsep}
\newenvironment{bibsection}%
        {\vspace{-\baselineskip}\begin{list}{}{%
       \setlength{\leftmargin}{\bibhang}%
       \setlength{\itemindent}{-\leftmargin}%
       \setlength{\itemsep}{\bibsep}%
       \setlength{\parsep}{\z@}%
        \setlength{\partopsep}{0pt}%
        \setlength{\topsep}{0pt}}}
        {\end{list}\vspace{-.6\baselineskip}}
\makeatother

% Layout: Puts the section titles on left side of page
\reversemarginpar

%
%         PAPER SIZE, PAGE NUMBER, AND DOCUMENT LAYOUT NOTES:
%
% The next \usepackage line changes the layout for CV style section
% headings as marginal notes. It also sets up the paper size as either
% letter or A4. By default, letter was used. If A4 paper is desired,
% comment out the letterpaper lines and uncomment the a4paper lines.
%
% As you can see, the margin widths and section title widths can be
% easily adjusted.
%
% ALSO: Notice that the includefoot option can be commented OUT in order
% to put the PAGE NUMBER *IN* the bottom margin. This will make the
% effective text area larger.
%
% IF YOU WISH TO REMOVE THE ``of LASTPAGE'' next to each page number,
% see the note about the +LP and -LP lines below. Comment out the +LP
% and uncomment the -LP.
%
% IF YOU WISH TO REMOVE PAGE NUMBERS, be sure that the includefoot line
% is uncommented and ALSO uncomment the \pagestyle{empty} a few lines
% below.
%

%% Use these lines for letter-sized paper
\usepackage[paper=letterpaper,
            %includefoot, % Uncomment to put page number above margin
            marginparwidth=1.2in,     % Length of section titles
            marginparsep=.05in,       % Space between titles and text
            margin=1in,               % 1 inch margins
            includemp]{geometry}

%% Use these lines for A4-sized paper
%\usepackage[paper=a4paper,
%            %includefoot, % Uncomment to put page number above margin
%            marginparwidth=30.5mm,    % Length of section titles
%            marginparsep=1.5mm,       % Space between titles and text
%            margin=25mm,              % 25mm margins
%            includemp]{geometry}

%% More layout: Get rid of indenting throughout entire document
\setlength{\parindent}{0in}

%% This gives us fun enumeration environments. compactitem will be nice.
\usepackage{paralist}

%% Reference the last page in the page numberd
%
% NOTE: comment the +LP line and uncomment the -LP line to have page
%       numbers without the ``of ##'' last page reference)
%
% NOTE: uncomment the \pagestyle{empty} line to get rid of all page
%       numbers (make sure includefoot is commented out above)
%
\usepackage{fancyhdr,lastpage}
\pagestyle{fancy}
%\pagestyle{empty}      % Uncomment this to get rid of page numbers
\fancyhf{}\renewcommand{\headrulewidth}{0pt}
\fancyfootoffset{\marginparsep+\marginparwidth}
\newlength{\footpageshift}
\setlength{\footpageshift}
          {0.5\textwidth+0.5\marginparsep+0.5\marginparwidth-2in}
\lfoot{\hspace{\footpageshift}%
       \parbox{4in}{\, \hfill %
                    \arabic{page} of \protect\pageref*{LastPage} % +LP
%                    \arabic{page}                               % -LP
                    \hfill \,}}

% Finally, give us PDF bookmarks
\usepackage{color,hyperref}
\definecolor{darkblue}{rgb}{0.0,0.0,0.3}
\hypersetup{colorlinks,breaklinks,
            linkcolor=darkblue,urlcolor=darkblue,
            anchorcolor=darkblue,citecolor=darkblue}

%%%%%%%%%%%%%%%%%%%%%%%% End Document Setup %%%%%%%%%%%%%%%%%%%%%%%%%%%%


%%%%%%%%%%%%%%%%%%%%%%%%%%% Helper Commands %%%%%%%%%%%%%%%%%%%%%%%%%%%%

% The title (name) with a horizontal rule under it
%
% Usage: \makeheading{name}
%
% Place at top of document. It should be the first thing.
\newcommand{\makeheading}[1]%
        {\hspace*{-\marginparsep minus \marginparwidth}%
         \begin{minipage}[t]{\textwidth+\marginparwidth+\marginparsep}%
                {\large \bfseries #1}\\[-0.15\baselineskip]%
                 \rule{\columnwidth}{1pt}%
         \end{minipage}}

% The section headings
%
% Usage: \section{section name}
%
% Follow this section IMMEDIATELY with the first line of the section
% text. Do not put whitespace in between. That is, do this:
%
%       \section{My Information}
%       Here is my information.
%
% and NOT this:
%
%       \section{My Information}
%
%       Here is my information.
%
% Otherwise the top of the section header will not line up with the top
% of the section. Of course, using a single comment character (%) on
% empty lines allows for the function of the first example with the
% readability of the second example.
\renewcommand{\section}[2]%
        {\pagebreak[3]\vspace{1.3\baselineskip}%
         \phantomsection\addcontentsline{toc}{section}{#1}%
         \hspace{0in}%
         \marginpar{
         \raggedright \scshape #1}#2}

% An itemize-style list with lots of space between items
\newenvironment{outerlist}[1][\enskip\textbullet]%
        {\begin{itemize}[#1]}{\end{itemize}%
         \vspace{-.6\baselineskip}}

% An environment IDENTICAL to outerlist that has better pre-list spacing
% when used as the first thing in a \section
\newenvironment{lonelist}[1][\enskip\textbullet]%
        {\vspace{-\baselineskip}\begin{list}{#1}{%
        \setlength{\partopsep}{0pt}%
        \setlength{\topsep}{0pt}}}
        {\end{list}\vspace{-.6\baselineskip}}

% An itemize-style list with little space between items
\newenvironment{innerlist}[1][\enskip\textbullet]%
        {\begin{compactitem}[#1]}{\end{compactitem}}

% An environment IDENTICAL to innerlist that has better pre-list spacing
% when used as the first thing in a \section
\newenvironment{loneinnerlist}[1][\enskip\textbullet]%
        {\vspace{-\baselineskip}\begin{compactitem}[#1]}
        {\end{compactitem}\vspace{-.6\baselineskip}}

% To add some paragraph space between lines.
% This also tells LaTeX to preferably break a page on one of these gaps
% if there is a needed pagebreak nearby.
\newcommand{\blankline}{\quad\pagebreak[2]}

% Uses hyperref to link DOI
\newcommand\doilink[1]{\href{http://dx.doi.org/#1}{#1}}
\newcommand\doi[1]{doi:\doilink{#1}}

% For \url{SOME_URL}, links SOME_URL to the url SOME_URL
\providecommand*\url[1]{\href{#1}{#1}}
% Same as above, but pretty-prints SOME_URL in teletype fixed-width font
\renewcommand*\url[1]{\href{#1}{\texttt{#1}}}

% For \email{ADDRESS}, links ADDRESS to the url mailto:ADDRESS
\providecommand*\email[1]{\href{mailto:#1}{#1}}
% Same as above, but pretty-prints ADDRESS in teletype fixed-width font
%\renewcommand*\email[1]{\href{mailto:#1}{\texttt{#1}}}

%%%%%%%%%%%%%%%%%%%%%%%% End Helper Commands %%%%%%%%%%%%%%%%%%%%%%%%%%%

%%%%%%%%%%%%%%%%%%%%%%%%% Begin CV Document %%%%%%%%%%%%%%%%%%%%%%%%%%%%

\begin{document}
\makeheading{Rahul Budhiraja}

\section{Objective}
%
A Prospective Master's Student who wishes to Pursue his Research interests in a Renowned Research Lab which utilizes and enhances his skills in solving challenging problems	


%\section{Security Clearance}
%
%Department of Defense Top Secret SCI with polygraph (expired: 2002)
%

%\section{Citizenship}
%
%INDIAN

\section{Education}
%
\href{http://www.iiita.ac.in/}{\textbf{Indian Institute of Information Technology}},
Allahabad,India

\begin{outerlist}

\item[] B.Tech,
             {Information Technology},
             July 2010
        \begin{innerlist}
        \item[-] CGPI:~8.4/10\\
	\item[-] Thesis Topic: \textbf {\emph{Tracking the Tracker}}: \emph{Investigating Tracking Methods in Augmented Reality from Fiducial Markers to Markerless Tracking Techniques}.\\
        
        \item[] Advisor:
                {Dr.Shekhar Verma}
        \item[] Area of Study: Augmented Reality,~Computer Vision
        \end{innerlist}

\end{outerlist}

\blankline

\href{http://www.mesqatar.org/}{\textbf{MES Indian School}}, Doha,Qatar

\begin{outerlist}

\item[] All India Senior School Certificate Examination,March 2006\\
        \href{http://www.cbse.nic.in/}
             {Central Board of Secondary Education}\\
	     {Percentage Obtained:}\textbf{92\%}
        %\end{innerlist}

\item[] All India Secondary School Examination,March 2004\\
        \href{http://www.cbse.nic.in/}
             {Central Board of Secondary Education}\\
	     {Percentage Obtained:}\textbf{85\%}
\end{outerlist}


\section{Research Experience}
%
\textbf{Research Assistant} \hfill {February 2009 to February 2010}
\begin{innerlist}

\item[] \href{http://www.mixedreality.nus.edu.sg/}{\textbf{Keio-NUS CUTE Center}},\href{http://www.osu.edu/}{Interactive and Digital Media Institute}
         \\National University of Singapore
\begin{innerlist}
\item[-] Augmented Reality in Military Applications (in Collaboration with \textbf{Ministry of Defence},Singapore) 
\end{innerlist}


\end{innerlist}

%%%%%%%%%%%%%%%%%%%%%%%%%%%%%%%%%%%%%%%%%%%%%%%%%%%%%%%%%%%%%%%%%%%%%%%%%

\blankline

\textbf{Research Assistant} \hfill {December 2009 to February 2010}
\begin{innerlist}

\item[] {\textbf{S}ignal,\textbf{I}mage and \textbf{L}anguage \textbf{P}rocessing Lab}\\
        {Indian Institute of Information Technology}
\begin{innerlist}
\item[-] A Preliminary Investigation of Augmented Reality Applications in Universities
\end{innerlist}
\end{innerlist}

%%%%%%%%%%%%%%%%%%%%%%%%%%%%%%%%%%%%%%%%%%%%%%%%%%%%%%%%%%%%%%%%%%%%%%%%%

\blankline

\textbf{Research Student} \hfill {July 2009 to November 2009}
\begin{innerlist}

\item[] \href{http://robita.iiita.ac.in/}{Robita Lab}\\
        \href{http://robita.iiita.ac.in/}{Indian Institute of Information Technology}
\begin{innerlist}
\item[-] {RoboCAM-A Multiclient Video Conferencing Tool} 

\end{innerlist}

\end{innerlist}

%%%%%%%%%%%%%%%%%%%%%%%%%%%%%%%%%%%%%%%%%%%%%%%%%%%%%%%%%%%%%%%%%%%%%%%%%

\blankline




%%%%%%%%%%%%%%%%%%%%%%%%%%%%%%%%%%%%%%%%%%%%%%%%%%%%%%%%%%%%%%%%%%%%%%%%%
\textbf{Research Student} \hfill {January 2009 to May 2009}
\begin{innerlist}

\item[] \href{http://prc.iiita.ac.in/}{\textbf{P}atent \textbf{R}eferral \textbf{C}enter},\\
        \href{http://prc.iiita.ac.in/}{Indian Institute of Information Technology}
\begin{innerlist}
\item[-] M2P:Many to PPT

\end{innerlist}
\end{innerlist}

%%%%%%%%%%%%%%%%%%%%%%%%%%%%%%%%%%%%%%%%%%%%%%%%%%%%%%%%%%%%%%%%%%%%%%%%%

% Add a little space to nudge next ``Conference Publications'' marginpar
% down to make room for tall ``Submitted Journal Publications''
% marginpar. If there are enough submitted journal publications, this
% space will not be needed (and should be removed).

\section{Research Interests}
%
Augmented Reality,Computer Vision,Human Computer Interaction,User Interface Design,Computational Photography


\vspace{0.1in}

\section{Conference Publications} \begin{bibsection}
\item Budhiraja.R,Verma.S,Pandey.A “P-SCAR:A Presentation System for Classrooms using Augmented
Reality” \emph{Proceedings of NICOGRAPH International:May 2010},Singapore.

\item Rahul Budhiraja, Shekhar Verma, and Arunanshu Pandey.Designing interactive presentation systems for classrooms. 
 in 
\emph{Proceedings of the 28th Annual International Conference on Design of Communication, SIGDOC 2010}
S{\~a}o Paulo, Brazil, September 26-29, 2010 pages: 259-260.

\end{bibsection}

\section{Other\\Publications} \begin{bibsection}
    \item Budhiraja,Rahul \textbf {{Tracking the Tracker}}: \emph{Investigating Tracking Methods in Augmented Reality from Fiducial Markers to Markerless Tracking Techniques}.
        Bachelor thesis, Indian Institute of Information Technology,
        Allahabad India, 2010.
    
    \item RoboCAM A Multiclient Video Conferencing Tool (Software Copyright Pending)
    \end{bibsection}

\section{Selected Projects}
%
  \textbf {Tracking System:An alternative approach}\\
 %\hfill Feb 2010 to Dec 2010\\

Some of the most successful AR tracking systems are those based on a combination of structure
from motion (SFM) and SLAM approaches, that both model and localize at the same time.
Our method, uses model building technology of Bundler to create various 3D maps of the environment.
These 3D maps are then used for AR localization over a wide area.
I worked on the Tracking and Rendering part of the project to implement our 
pose estimation algorithm and rendering both 3D and 2D Augmentations with the help of the system.
\\

\textbf {Advisor}:~Prof Adrian David Cheok,~Prof Gerhard Roth \hfill\textbf{ [Feb'10 to Dec'10 ] }

\blankline

\textbf {Training Editor:An AR Authoring System}\\% \hfill dsdsdsdsdsd \\
 
Another module of our project was to create an AR Authoring system which could position objects in a Virtual World relative to a Map and these transformations would be 
registered accurately with the Tracking System.For this application,we used Lib3DS to load the .3DS model,TinyXML to store \& transmit the XML Stream,OpenGL to build the GUI and LibSDL 
to place the textures on the 3DS Models.The Maps were created offline and stored in the PLY format.
\\ 

\textbf {Advisor}:~Prof Adrian David Cheok,~Prof Gerhard Roth \hfill\textbf{ [Dec'10 to Feb'11 ] }


 \blankline

\textbf {P-SCAR:A Presentation System for Classrooms using Augmented Reality} \\\\
    In this project,we developed a presentation system for classrooms in which the teacher could give interactive presentations to students present within the classroom and those at home
.At the teacher's end ,the teacher controls the flow of the presentation by means of Speech and Hand Gestures and accordingly these events are reflected onto the students application.The Entire Presentation is divided into sessions such as Teaching,Q \& A etc. which are managed by the system. 
\\

\textbf {Advisor}:~Prof.Shekhar Verma \hfill\textbf{ [Dec'09 to Mar'10 ]}

\blankline 

\textbf {A Prelim Investigation of Augmented Reality Applications in Universities}\\\\
In this project,we had to demonstrate the use of Marker Based AR in building simple interactive applications useful for a university.Possible Applications could be useful to either the faculty,staff or even university visitors 
.The Applications developed included a Business Card for staff,An Interactive 3D Viewer to show to visitors and Displaying 3D Objects on Markers as a supplement to classroom presentations(for courses like Computer Graphics and Image Processing).\\

\textbf {Advisor}:~Prof.Shekhar Verma \hfill\textbf{ [Nov'09 to Jan'10 ]}

%\textbf {RoboCAM} \hfill dsdsdsdsdsd 
  
%The project aims to build a A multiclient video conferencing tool using the Flash media server
%streaming capabilities and also providing facility of text chat .The software can also be used for
%multicasting in addition to a secure user credential management storage.\\

%\textbf {Advisor}:~Prof.G.C.Nandi 


\section{Freelance\\Activities}
%
\textbf{AR Related}

\begin{outerlist}

\item Developed an Augmented Reality Application using FLARToolkit for a student of
Advertising and Multimedia in the University of Kairouan,Tunisia.
\blankline
\item Built a Framework to dynamically link AR Markers and Collada Objects using a SQL Database for a student of North Sumatara University,Indonesia.
\blankline
\item Collaborated with a Uruguayan Video Producer and Web Artist to showcase his work using AR Markers.Work involved creating a Framework to showcase his videos on AR Markers and provide facility to easily add/delete videos and markers.
\blankline

\item Developed a FLARToolkit Application for a student of Universidade Potiguar,Brazil to illustrate the usage of FLARToolkit .The Client had experience of creating ARToolkit appplications and wanted a sample application using Actionscript for developing web-based AR applications. 
\blankline

\item Created a FLARToolkit application for a Smash Magazine personnel to demonstrate the usage of Augmented Reality in giving Interactive Presentations.
\blankline

\end{outerlist}

\blankline

\textbf{Visual FX}
  
\begin{outerlist}
\item Developed a Promotional Video for \textbf{Aphelion(IV)} ,a track from \textit{The Palatin Project}. 
Worked with an Electronic Music Composer to use raw footage from a Temperate Rainforest to create the video using Visual FX Software.
\end{outerlist}
\blankline

\section{Technical \\Skills}
%
\textbf {Languages}: C, C++, SQL, HTML, XML,VB.NET,ASP.NET,ActionScript,Matlab,
OpenGL,\LaTeX,Bash Script,PHP
  
\blankline 

\textbf {Platforms}:SuSE 8.1 to 10.0, Windows 98/NT/XP/Vista,Ubuntu 9.1 to 10.10,Android

\blankline

\textbf {Software Experience}

\blankline 

\begin{innerlist}[\enskip\textbullet] 


\item[] \textbf{IDEs}:Adobe Flash CS4,Oracle 9i,Visual Studio,Eclipse,GNU Emacs


\item[] \textbf{3D Rendering}: Poser Pro 7,Blender

\item[] \textbf{Video Editing}: Cyberlink Power Director, Adobe After Effects CS3

\item[] \textbf{Sound Editing}: Adobe Soundbooth CS3,Adobe Audition CS3

\item[] \textbf{Image Editing}: Adobe Photoshop CS4,Corel Paintshop Pro 8

\end{innerlist}



%\section{Papers in Preparation} \begin{bibsection}
%    \item Pavlic, T.P., K.M.~Passino. Distributed optimization under
%        constraints: Pareto-optimal intelligent lighting.
%
%    \item Pavlic, T.P. The ideal free distribution as degenerate form of
%        nutrient-constrained optimization.
%\end{bibsection}


\section{Honors and Awards}
%
-\textbf{Runner-Up}\textit{(Phase I)}, ~Topcoder-Alcatel Lucent 100 Apps in 100 days 

\blankline

-\textbf{Meritorious Student Award },in recognition for Outstanding Academic performance,2006. 

\blankline

-\textbf{Principal's Honors List}-2003

\blankline

-\textbf{Sixth Position}.Gulf Chemistry Olympiad 

\blankline

-\textbf{Best IT Project},MES Computer Science Department 2003.

\blankline

-\textbf{First class honours},ASP.NET Course.
%-Won prizes in Dance Performances in the college(Second in 2007,2009 and Third in 2006)
%\blankline
%-\textbf {Lead singer},L.O.C.A IIITA Metal Band (Jul’09-Nov’09)
\blankline

\section{Positions of Responsibility}
%
\textbf{Alumni Member}:ACM IIITA Chapter and North India SIGCHI Chapter

\blankline

\textbf{Head Prefect},MES Indian School(April 2005 - March 2006)\\
\textbf{Assistant Head Prefect},MES Indian School(April 2004 - March 2005)

\blankline

\textbf{Lead Designer },Effervescence Promotional Video Team (Jan’08 -May’08).

\blankline

%\textbf{Treasurer },US	HMA-Dance Club of IIITA (Jan ’07 - Dec ’07).
%\blankline

\textbf{Member}.Accommodation Committee of the 2nd Science Conclave 2009,IIIT Allahabad

\blankline

\textbf{Student-Industry Contact} for Campus Placements.

\blankline

%\textbf{Captain} :College Tennis Team(2007 - 2009) \& School Tennis Team(2002-2006).
%\blankline

\textbf{Chief Co-ordinator} Gamers Asylum: Counterstrike event in Effervescence MM’8

\section{Industrial Experience}
%
\href{http://www.tcs.com/}{\textbf{Tata Consultancy Services}},
Gurgaon,Haryana India
\begin{outerlist}

\item[] \textit{Product Demonstrations Designer}%
        \hfill \textbf{May 2008 to July 2008}
\begin{innerlist}	
\item Developed demonstrational videos for four Multimedia Applications of the IP Multimedia Subsystem(IMS) group of the Wireless Telecom Applications Department.
\item These Videos will supplement their Live Demonstrations at International Telecom Product Expos.
\end{innerlist}

\end{outerlist}

\blankline

\href{http://www.thesportscampus.com/}{\textbf{The Sports Campus.com}}

\begin{outerlist}

\item[] \textit{Research Analyst}%
        \hfill \textbf{December 2009 to January 2010}

\begin{innerlist}
\item Research and Compile in-depth articles and sports Trivia for the company.

\end{innerlist}

\end{outerlist}

\section{References}
%
\href
{http://www.adriancheok.info}
{\textbf{Dr.~Adrian Cheok}}
(e-mail:~\href{mailto:adriancheok@mixedrealitylab.org}{adriancheok@mixedrealitylab.org})
%
\begin{innerlist}
    \item Professor \\
        \href{http://www.kmd.keio.ac.jp/en/}{Graduate School of Media Design},\href{http://www.osu.edu/}{~Keio University}
    \item Associate Professor,\\
        \href{http://www.idmi.nus.edu.sg/}{Interactive and Digital Media Institute},
        \href{http://www.nus.edu.sg/}{National University of Singapore}
         
    \item[$\star$] \emph{Dr.~Cheok was the PI in our Project with the Ministry of Defence and is the director of the CUTE Center.}
\end{innerlist}

\blankline

\href
{http://people.scs.carleton.ca/~roth/}
{\textbf{Dr.~Gerhard Roth}}
(e-mail:~\href{mailto:gerhardroth@rogers.com}{gerhardroth@rogers.com}; phone:~(613)993-1219)
\begin{innerlist}
    \item Adjunct Research Professor\\
        {School of Computer Science},~
        {Carleton Univesity}

    \item[$\star$] \emph{Dr.~Roth is a Co-PI who designed and supervised my work during the development of our Tracking and Rendering System and gave me useful education and advice on Augmented Reality and Computer Vision}
\end{innerlist}

\blankline

{\textbf{Dr.~Shekhar Verma}}
(e-mail:~\href{mailto:sverma@iiita.ac.in}{sverma@iiita.ac.il
'
dsn}) 
%
\begin{innerlist}
    \item Associate Professor,
        {Signal,Image and Language Processing Lab}\\
        {Indian Institute of Information Technology}

        \item[$\star$] \emph{Dr.~Verma was my bachelor thesis advisor and supervised my work at the SILP Lab}
\end{innerlist}


\blankline


{\textbf{Dr.~Hideaki Nii}}
(e-mail:~\href{mailto:nii@mixedrealitylab.org}{nii@mixedrealitylab.org})
%
\begin{innerlist}
    \item Research fellow,
        {Keio-NUS CUTE Center}
        \\
        {Interactive and Digital Media Institute,National University of Singapore}


    \item[$\star$] \emph{Dr.~Nii is a co-PI for the MINDEF project and supervised my work in developing an application for the CBRE group during the MINDEF Singapore Project}
\end{innerlist}

\blankline


{\textbf{Dr.~R.C Tripathi}}
(e-mail:~\href{mailto:rct@iiita.ac.in}{rct@iiita.ac.in})
\begin{innerlist}
    \item Professor,Dean (R\&D)\\
        \href{http://www.iiita.ac.in/}{Indian Institute of Information Technology}

       \item[$\star$] \emph{Dr.Tripathi has been my supervisor for two of my projects at the Patent Referral Center in my University}
 \end{innerlist}

\blankline

\section{Contact Information}
%
% NOTE: Mind where the & separators and \\ breaks are in the following
%       table.
%
% ALSO: \rcollength is the width of the right column of the table
%       (adjust it to your liking; default is 1.85in).
%
\newlength{\rcollength}\setlength{\rcollength}{1.85in}%
%
\begin{tabular}[t]{@{}p{\textwidth-\rcollength}p{\rcollength}}
%\href{http://www.cse.osu.edu/}%
 %    {Department of Computer Science and Engineering} & \\
%\href{http://www.osu.edu/}{The Ohio State University}
{HP-22 Maurya Enclave}                           & \textit{Mobile:} +974-44772802\\
{Pitampura New Delhi}&\textit{Mail:}\email{rahul.budhiraja.dark@gmail.com}\\  
{India-110034}     & \textit{WWW:}\href{http://www.rahulbudhiraja.com/}{www.rahulbudhiraja.com}\\       
    
\end{tabular}

%\begin{center}
 %\href{http://www.picasaweb.google.com/rahul.budhiraja.dark }{\includegraphics{./Icons_DefaultSize/twittericon.png}}
 % t.png: 32x32 pixel, 72dpi, 1.13x1.13 cm, bb=0 0 32 32
  %\includegraphics{./Icons/Picasa.png}
 %\includegraphics{./Icons/Facebook-icon.png}
 %\includegraphics{./Icons/youtube_icon.png}
%\end{center}


\end{document}
